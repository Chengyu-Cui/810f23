\documentclass[12pt]{article}
\usepackage{fullpage,hyperref}\setlength{\parskip}{3mm}\setlength{\parindent}{0mm}
\begin{document}

\begin{center}\bf
Homework 7. Due by 11:59pm on Sunday 10/22.

Collaborative research \& Human participants and animal subjects

\end{center}
Most statisticians do not have to worry directly about running large scientific research groups and dealing with the bureaucracy of ethical data collection. Yet, most statisticians collaborate sometimes with scientists who do. Read pages 24--28, 39--42 and 48--49 of {\em On Being a Scientist}.  Write brief answers to the following questions, by editing the tex file available at \url{https://github.com/ionides/810f23}, and submit the resulting pdf file via Canvas.

\begin{enumerate}

\item What is an IRB? Does a project studying aggregated observational data on human subjects (say, the total number of road accident injuries per state per year) need IRB approval to receive federal funding?

YOUR ANSWER HERE

\item Suggest some ingredients which could lead to successful collaboration between two statisticians and/or between a statistician and a scientist.

YOUR ANSWER HERE

\item Collaborative group sizes can be small or large. Identify some strengths and weaknesses of larger collaborative groups relative to smaller collaborative groups.

YOUR ANSWER HERE
 
\item \label{p1} What are the advantages and disadvantages of being a conscientious collaborator who makes careful, thoughtful but timely contributions to the project, reads widely and takes the time to understand as much of the project as possible.

YOUR ANSWER HERE

\item Would you expect a PhD thesis adviser to act like the conscientious collaborator of question~\ref{p1} on your own thesis research? 

YOUR ANSWER HERE

\item What are some advantages and disadvantages of joining a project and then making a minimal contribution? Can this be responsible behavior? Consider the following example: you help a scientist carry out a statistical procedure and you help write up the paragraph describing it; you accept coauthorship on the resulting paper, without spending time on all other aspects of the paper.

YOUR ANSWER HERE

\item How can one maintain a reasonable level of agreement within a collaboratoration on the expected involvement of each collaborator?

YOUR ANSWER HERE

\end{enumerate}
\end{document}
